\documentclass[11pt]{book}
\usepackage[utf8]{inputenc}
\usepackage{amsmath,amssymb}
\usepackage{hyperref}
\usepackage{graphicx}
\usepackage{geometry}
\geometry{margin=1in}

\title{Mathematical Encyclopedia \\ \large Volume 4 (Part 4)}
\author{Compiled Reference Work}
\date{}

\begin{document}
\maketitle

\begin{center}
\textbf{Comprehensive Mathematics Reference Series}\\
\textit{Volume 4 (Part 4)}
\end{center}

\cleardoublepage
\tableofcontents
\cleardoublepage

\chapter*{Preface}
This fourth installment of Volume 4 continues the encyclopedia with entries in the range of \textbf{T} topics, expanding on trigonometric, geometric, and analytic concepts such as Taylor expansions, tensors, tetrahedra, and theta functions. The consistent notation and concise exposition are intended for advanced students, educators, and researchers.

\chapter{T Topics (Continuation)}
\section{Tau Function}
The \textbf{tau function} appears in number theory as the Ramanujan tau function $\tau(n)$, defined by the generating function:
\begin{equation}
\sum_{n=1}^{\infty} \tau(n) q^n = q \prod_{m=1}^{\infty} (1 - q^m)^{24}.
\end{equation}
It encodes deep arithmetic information about modular forms.

\section{Taylor Polynomial}
The $n$th-degree \textbf{Taylor polynomial} for a function $f(x)$ about $x = a$ is:
\begin{equation}
P_n(x) = \sum_{k=0}^{n} \frac{f^{(k)}(a)}{k!}(x-a)^k.
\end{equation}
This polynomial approximates $f(x)$ near $x = a$.

\section{Tensor}
A \textbf{tensor} is a multidimensional array generalizing scalars, vectors, and matrices. A rank-2 tensor can be represented as $T_{ij}$. Tensor calculus underpins physics, particularly in general relativity.

\section{Tetrahedron}
A \textbf{tetrahedron} is a polyhedron with four triangular faces. The volume is given by:
\begin{equation}
V = \frac{1}{3} A h,
\end{equation}
where $A$ is the area of the base and $h$ is the height.

\section{Theta Function}
The \textbf{Jacobi theta function} is defined as:
\begin{equation}
\theta(z, q) = \sum_{n=-\infty}^{\infty} q^{n^2} e^{2\pi i n z}.
\end{equation}
It plays an important role in complex analysis and number theory.

\section{Theorem}
A \textbf{theorem} is a statement proven to be true using logical reasoning based on axioms and prior results.

\section{Three-Dimensional Geometry}
In 3D geometry, points are represented by ordered triples $(x, y, z)$. The distance between points $A(x_1, y_1, z_1)$ and $B(x_2, y_2, z_2)$ is:
\begin{equation}
AB = \sqrt{(x_2 - x_1)^2 + (y_2 - y_1)^2 + (z_2 - z_1)^2}.
\end{equation}

\section{Torus}
A \textbf{torus} is the surface formed by revolving a circle in three-dimensional space about an axis coplanar with the circle. Its equation in Cartesian coordinates is:
\begin{equation}
\left(\sqrt{x^2 + y^2} - R\right)^2 + z^2 = r^2,
\end{equation}
where $R$ is the distance from the center of the tube to the center of the torus, and $r$ is the radius of the tube.

\section{Transformation Matrix}
A \textbf{transformation matrix} represents linear transformations such as rotation, scaling, and reflection in vector spaces. For example, a 2D rotation by angle $\theta$ is:
\begin{equation}
R(\theta) = 
\begin{pmatrix}
\cos \theta & -\sin \theta \\
\sin \theta & \cos \theta
\end{pmatrix}.
\end{equation}

\section{Transpose of a Matrix}
For a matrix $A = [a_{ij}]$, the \textbf{transpose} $A^T$ is obtained by interchanging rows and columns:
\begin{equation}
(A^T)_{ij} = a_{ji}.
\end{equation}

\section{Triangle Inequality}
For real numbers or vectors:
\begin{equation}
|a + b| \leq |a| + |b|.
\end{equation}
In Euclidean space, the length of one side of a triangle is always less than or equal to the sum of the lengths of the other two sides.

\section{Truncated Cone}
A \textbf{truncated cone} (frustum) has volume:
\begin{equation}
V = \frac{1}{3} \pi h (R^2 + Rr + r^2),
\end{equation}
where $R$ and $r$ are the radii of the two circular faces, and $h$ is the height.

\section{Turing Machine}
A \textbf{Turing machine} is a theoretical computational model with an infinite tape, a read/write head, and a set of states. It formalizes the concept of algorithmic computation.

\chapter*{Notes on Conversion}
This \LaTeX{} document maintains the uniform formatting of the previous encyclopedia volumes and contains mathematical definitions, formulas, and conceptual summaries for the final \textbf{T} entries.\\[1em]
If you upload Volume 5 or subsequent sections, I’ll continue formatting them identically.

\end{document}
