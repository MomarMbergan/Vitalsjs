\documentclass[11pt]{book}
\usepackage[utf8]{inputenc}
\usepackage{amsmath,amssymb}
\usepackage{hyperref}
\usepackage{graphicx}
\usepackage{geometry}
\geometry{margin=1in}

\title{Mathematical Encyclopedia \\ \large Volume 4 (Part 3)}
\author{Compiled Reference Work}
\date{}

\begin{document}
\maketitle

\begin{center}
\textbf{Comprehensive Mathematics Reference Series}\\
\textit{Volume 4 (Part 3)}
\end{center}

\cleardoublepage
\tableofcontents
\cleardoublepage

\chapter*{Preface}
This third installment of Volume 4 continues the encyclopedia’s coverage of mathematical concepts from \textbf{S} to \textbf{T}. It includes key entries on derivatives, trigonometric functions, Taylor series, topology, and transformations. Each entry retains rigorous yet concise explanations suitable for both study and quick reference.

\chapter{S Topics}
\section{Second Derivative}
The \textbf{second derivative} of a function $f(x)$ measures the rate of change of its first derivative:
\begin{equation}
f''(x) = \frac{d^2 f}{dx^2}.
\end{equation}
It provides information about concavity: if $f''(x) > 0$, $f$ is concave up; if $f''(x) < 0$, $f$ is concave down.

\section{Sequence Convergence}
A sequence $\{a_n\}$ converges to a limit $L$ if, for every $\epsilon > 0$, there exists $N$ such that:
\begin{equation}
|a_n - L| < \epsilon \quad \text{for all } n > N.
\end{equation}

\section{Series}
A \textbf{series} is the sum of the terms of a sequence:
\begin{equation}
\sum_{n=1}^{\infty} a_n.
\end{equation}
Convergence tests such as ratio, root, and comparison tests determine whether a series converges.

\section{Set}
A \textbf{set} is a collection of distinct objects, written as $A = \{a_1, a_2, a_3, \ldots\}$. Operations include union ($A \cup B$), intersection ($A \cap B$), and complement ($A'$).

\section{Set Theory}
\textbf{Set theory} forms the foundation of modern mathematics, defining relations, functions, and structures using formal logic and element membership.

\section{Sigma Notation}
The Greek letter $\Sigma$ represents summation:
\begin{equation}
\sum_{i=1}^{n} a_i = a_1 + a_2 + \cdots + a_n.
\end{equation}

\section{Sine Function}
The \textbf{sine} of an angle $\theta$ in a right triangle is:
\begin{equation}
\sin \theta = \frac{\text{opposite side}}{\text{hypotenuse}}.
\end{equation}
Its derivative is $\frac{d}{dx}\sin x = \cos x$.

\section{Slope}
The \textbf{slope} of a line measures its steepness:
\begin{equation}
m = \frac{y_2 - y_1}{x_2 - x_1}.
\end{equation}

\section{Sphere}
A \textbf{sphere} in 3D space centered at $(x_0, y_0, z_0)$ with radius $r$ satisfies:
\begin{equation}
(x - x_0)^2 + (y - y_0)^2 + (z - z_0)^2 = r^2.
\end{equation}

\chapter{T Topics}
\section{Tangent Function}
The \textbf{tangent} of an angle $\theta$ is:
\begin{equation}
\tan \theta = \frac{\sin \theta}{\cos \theta}.
\end{equation}
Its derivative is $\frac{d}{dx}\tan x = \sec^2 x$.

\section{Taylor Series}
The \textbf{Taylor series} of a function $f(x)$ about $x = a$ is:
\begin{equation}
f(x) = f(a) + f'(a)(x-a) + \frac{f''(a)}{2!}(x-a)^2 + \cdots.
\end{equation}
If $a = 0$, it is called the \textit{Maclaurin series}.

\section{Topology}
\textbf{Topology} studies properties preserved under continuous deformation. A topological space $(X, \tau)$ consists of a set $X$ and a collection $\tau$ of open subsets satisfying specific axioms.

\section{Transformation}
A \textbf{transformation} maps points or functions from one form to another. Linear transformations preserve vector addition and scalar multiplication:
\begin{equation}
T(\vec{u} + \vec{v}) = T(\vec{u}) + T(\vec{v}), \quad T(c\vec{u}) = cT(\vec{u}).
\end{equation}

\section{Transcendental Equation}
A \textbf{transcendental equation} involves transcendental functions (e.g., exponential, logarithmic, trigonometric):
\begin{equation}
e^x = x + 2.
\end{equation}
Such equations generally lack closed-form solutions and are solved numerically.

\section{Triangle}
A \textbf{triangle} is a polygon with three sides. The area $A$ is given by:
\begin{equation}
A = \frac{1}{2} b h.
\end{equation}

\section{Trigonometric Identities}
Fundamental trigonometric identities include:
\begin{align}
\sin^2 \theta + \cos^2 \theta &= 1, \\
1 + \tan^2 \theta &= \sec^2 \theta, \\
1 + \cot^2 \theta &= \csc^2 \theta.
\end{align}

\section{Trigonometric Functions}
Trigonometric functions relate angles to ratios of sides in right triangles and extend periodically to all real numbers.

\chapter*{Notes on Conversion}
This LaTeX document maintains the consistent design and structure of earlier encyclopedia volumes. It includes definitions, equations, and formatted examples for key mathematical terms beginning with \textbf{S} and \textbf{T}.\\[1em]
If you upload the next encyclopedia volume or continuation, I’ll format it identically for a seamless series.

\end{document}
