\documentclass[11pt]{book}
\usepackage[utf8]{inputenc}
\usepackage{amsmath,amssymb}
\usepackage{hyperref}
\usepackage{graphicx}
\usepackage{geometry}
\geometry{margin=1in}

\title{Biomedical Signal and Image Processing \\ \large Second Edition}
\author{Kayvan Najarian \\ Robert Splinter}
\date{2012}

\begin{document}
\maketitle

\begin{center}
\textbf{CRC Press / Taylor \& Francis Group}\\
ISBN: 978-1-4398-7033-4
\end{center}

\cleardoublepage
\tableofcontents
\cleardoublepage

\chapter*{Preface}
The first edition of the book \textit{Biomedical Signal and Image Processing} was published by CRC Press in 2005. It was used by many universities and educational institutions as a textbook for upper undergraduate and first-year graduate level courses in signal and image processing. It was also used by companies and research institutions as a reference book for their projects. This highly encouraging impact of the first edition motivated us to improve and produce a second edition.

The following improvements have been made to this edition:
\begin{itemize}
    \item Editorial corrections for typos, grammatical errors, and ambiguities.
    \item Additional MATLAB\textregistered\ examples to illustrate core concepts.
    \item Expanded explanations for complex signal and image processing ideas.
\end{itemize}

Kayvan Najarian \\
Virginia Commonwealth University, Richmond, VA

\chapter*{Acknowledgments}
Dr. Najarian thanks Dr. Joo Heon Shin for invaluable and detailed feedback. He is especially grateful to Dr. Abed Al Raoof Bsoul, whose diligence and insight contributed extensively to this edition. Further thanks to Paul Junor, Dr. Sharam Shirani, Alireza Darvish, and Jerry James Zacharias for their input on various chapters. Their feedback improved this book's clarity and presentation.

\chapter*{Introduction}
\section*{I.1 Processing of Biomedical Data}
Processing of biological and medical information has long been a dynamic field of life sciences. Prior to the use of digital computers, analysis relied on human expertise and heuristic judgment. The accuracy of such manual diagnostics was limited by human error, fatigue, and subjectivity. Advances in imaging modalities such as MRI, CT, and ultrasound now generate data volumes requiring computational processing.

Biomedical information processing employs mathematical tools to extract diagnostic insights from data. The process includes sensing, filtering, feature extraction, and classification. Figure~\ref{fig:biomedical_block} shows a general flow.
\begin{figure}[h]
\centering
\rule{0.6\textwidth}{0.3\textwidth}
\caption{Block diagram of a typical biomedical signal/image processing system. (Placeholder)}
\label{fig:biomedical_block}
\end{figure}

\section*{I.2 About the Book}
This book may serve as a senior undergraduate or first-year graduate text. It assumes basic calculus and familiarity with complex variables; knowledge of linear algebra is helpful. It details signal and image processing mathematics and provides numerous MATLAB\textregistered\ examples and exercises.

\part{Introduction to Digital Signal and Image Processing}

\chapter{Signals and Biomedical Signal Processing}
\section{1.1 Introduction and Overview}
The concept of a “signal” underlies all topics in this book. Signals represent ordered sequences describing variations of quantities, such as body temperature over time.

\section{1.2 What Is a Signal?}
A one-dimensional (1-D) signal is an ordered sequence of numbers. Examples include ECG (electrocardiogram) and EEG (electroencephalogram). Images are two-dimensional (2-D) signals.

\section{1.3 Analog, Discrete, and Digital Signals}
Signals are categorized based on continuity:
\begin{itemize}
\item \textbf{Analog signals}: Continuous in time and amplitude.
\item \textbf{Discrete signals}: Continuous in amplitude, discrete in time.
\item \textbf{Digital signals}: Discrete in both time and amplitude.
\end{itemize}
Sampling and quantization introduce digital signals suitable for computer processing.

\section{1.4 Processing and Transformation of Signals}
Signal transformations such as the Fourier Transform (FT) and Wavelet Transform reveal characteristics in time and frequency domains.

\section{1.5 Signal Processing for Feature Extraction}
Processed signals yield diagnostic features, e.g., tumor detection from CT or ECG rhythm classification.

\section{1.6 Characteristics of Digital Images}
Digital images are 2-D signals capturing intensity and spatial information. They can be grayscale or color, with pixel intensities typically ranging 0–255.

\chapter*{Notes on Conversion}
This LaTeX file is a standalone reconstruction of the book’s structure from your uploaded PDF. It includes:
\begin{itemize}
\item Preface, Acknowledgments, Introduction, and Chapter 1 (partial text).
\item Clean LaTeX book class formatting.
\item Placeholders for images, equations, and further chapters.
\item Sections and subsections prepared for content expansion.
\end{itemize}

\noindent
If you’d like, I can continue converting the rest of this PDF (Chapters 2–18) into this same \LaTeX{} file, maintaining book structure.

\end{document}
